\section{Conclusion}
We proposed a new authentication scheme called the evolving passwords based on physical controls for WLAN authentication. WLAN passwords automatically evolve at the predetermined time. A random number called physical parameter is used to update passwords which can only be obtained in a specific location protected by physical controls. Once a password evolves, users must pass the physical access control, enter the specific location, and get the physical parameter. Only users calculate out the new passwords,  can they access WLAN again. A guest finishing his/her visit cannot pass the physical access control, so he/she cannot get the new password and access WLAN again. 


A WLAN system deployed with the proposed authentication scheme consists of three parts: a master AP which can updates its own password independently, several slave APs which should interact with the master AP to update their own passwords, and many mobile devices which can automatically update their own passwords and access WLAN. At the predetermined update time, the master AP will generate a random number as a new physical parameter, calculate a new password using the old password and the new physical parameter. Then the master AP will broadcast the new physical parameter to the specific location protected by physical access controls and use the new password to authenticate mobile devices. At the same time, the slave APs will require new passwords from the master AP and then use the new password to authenticate mobile devices. As for mobile devices, they should also update their passwords by the same way as APs. In order to update their own passwords, mobile devices should enter the specific location and capture the physical parameter. 


The proposed authentication scheme can enhance WLAN security. Due to the periodic change of passwords, the possibility of unauthorized persons accessing the network by guessing passwords is reduced and prevent fake AP from setting the same password to induce users to access the insecure network. At the same time, the scheme achieve fine-grained access control for users.If the mobile devices password is not updated synchronously after the AP-side password is updated, when the user enters the network area again, the old password cannot be used to access the current network.In a place with a large number of guest users, this provides a convenient way to dynamically revoke user access rights. What’s more, the proposed authentication scheme can enhance WLAN security and  has a low impact on users’ experience at the same time. WLAN passwords automatically update without participating of users and administrators. All APs in the WLAN system will update their passwords in a very short time and passwords evolving have little influence on mobile devices accessing WLAN. The time mobile devices takes to update password is negligible enough compared with the time it takes to accessing WLAN. 
