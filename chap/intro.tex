\section{Introduction}
\label{sec:Introduction}
Nowadays, WLAN has gained much more popularity than ever before with the popularity of mobile devices and mobile Internet because of its cheapness and fastness. We can now receive wireless signal almost everywhere such as in a park, on a bus, at home, or in workplace~\cite{gong2015advanced}. When we visit a new place, requiring for WLAN access often becomes one of the first things we do. Hence, it is of paramount importance to improve wireless communications security because more and more people are using wireless networks for online banking and personal emails, owing to the widespread use of smartphones~\cite{zou2016survey}. At the same time, there are two kinds of users joining such WLAN with different demands. The staff want a long-term WLAN access authorization while visitors only need to temporarily join WLAN, so  WLAN authentication becomes a problem.  It should achieve differential WLAN access control for both visitors and staff, and dynamic WLAN access control for different visitors. 


There are several solutions to settle above problems now, an  WLAN can be secured by the 802.11i protocol with pre-shared key mode(also known as WPA-PSK protocol)~\cite{Gu2011Research}. To provide higher security and more fine-grained access control, an organization can build a robust and secure  WLAN by the 802.11i protocol with 802.1X authentication server mode (also known as WPA-EAP protocol)~\cite{lashkari2009survey}. However, the most common authentication mechanism of a  WLAN is to deploy a web portal server equipped with WIFIDog. The web portal server can use various ways to authenticate users. 


One solution is to build an open WLAN. There is no authentication of this kind of WLAN, The WLAN is not protected by password, backward authentication server or any other approaches.It is the most convenient solution, but the disadvantage is also obvious. Any visitors can join WLAN freely including unauthorized users, more than that,this solution don't provide secure communication and encryption transmission~\cite{ Park2003WLAN}. Attackers will be able to intercept authorized users’ communication data, so users' private information may be revealed~\cite{Lamport1981Password}. 


Another solution is to build a relatively secure WLAN by the 802.11i protocol with pre-shared key mode(also known as WPA-PSK protocol)~\cite{macmichael2005auditing}. This password-based authentication mechanism~\cite{Burch2016Time} prevents unauthorized users who do not know the password accessing WLAN. If visitors want to join WLAN, they need to ask for password from administrators or authorized users first. However, this kind of authentication cannot prevent users who have known password joining WLAN as password usually stay unchanged for a long time. Thus, it cannot revoke visitors’ authorization of WLAN access when they leave. Besides, password is usually easy to guess, and a brute force attack is considered effective. If the password is compromised, there is no security for this kind of authentication. 


A more fine-grained but uncommon solution is to build a WLAN by the 802.11i protocol with 802.1X authentication server mode (also known as WPA-EAP protocol)~\cite{akhlaq2007comparative}. Administrators create accounts for every visitor, and probably set expiration time at the same time. When they leave, administrators delete their accounts to revoke their authorization. Users join WLAN with their own accounts during expiration time. But this kind of authentication adds a heavy burden on administrators especially when there are quite a few visitors coming and leaving everyday because everything relies on administrators. For convenience, administrators may create a single account for all visitors. All visitors share the same account which remains unchanged for a long time. Just like password-based authentication, it is hard for administrators to revoke authorization when visitors leave away. For educational institutions like universities, they have another choice - joining the Eduroam alliance and building a guest WLAN for visitors from other members in the alliance. Eduroam ~\cite{florio2005eduroam}provides wireless roaming for students, teachers, and researchers, enabling these people to join other educational institutions’ WLAN using accounts in their own institutions. So other educational institutions need not create accounts for these people. Eduroam helps WLAN reduce its burden in maintaining visitors’ accounts. But not all visitors are able to enjoy the wireless roaming service, administrators should also consider visitors not in the Eduroam alliance~\cite{Ó2007Deploying}. 


A portal server equipped with WiFiDog is widely used in guest WLAN~\cite{Lenczner2005Wireless}. Exactly speaking, it is a way of WLAN authorization rather than a way of WLAN authentication. Mobile devices should authenticate to APs first by the above mentioned ways. In most cases, the WLAN is open. That is, all users (authorized or unauthorized) are allowed to join WLAN. However, users cannot surf the Internet until they pass the authentication of the portal server. Once users connect to WLAN, they will be redirected to a login web page and required for login. There are several methods to login: 
\begin{itemize}
    \item Login by telephone number. Users type in their telephone numbers first. The portal server then will send a message carrying a verification code to the user’s telephone. Users then type in the verification code to login~\cite{Hallsteinsen2007Using}; 
    \item Login by wechat account. Users press the key of wechat login. This action will evoke the wechat application. In the wechat applicant users press the key of consent. Then users will be redirected back to the login page and achieve single sign-on; 
    \item Login by wechat friends. The process of this login method is just like the above. However, the login user is allowed to achieve single sign-on only if the his/her wechat account is the administrator’s wechat account;
     \item 	Login by inviters’ invitation. A QR code is displayed on the login page waiting for inviters (aka authorized users) scanning. Inviters scan the QR code to let invitees login to portal server;
     \item Login by username and password. Users type in their usernames and passwords at the login page to login.
\end{itemize}


Though there are so many authentication methods can be applied to portal server, there are still potential security risks on this kind of authentication. This kind of authentication does not provide extra security enhancement on WLAN transmission. In most cases, WLAN is open, all datas will be transmitted in plain text on WLAN. Besides, the former two authentication methods cannot filter users at all. All users including authorized or unauthorized can join WLAN freely. The last authentication method put heavy burden on administrators as it is a centralized solution.  

However, none of the existing solutions can realize the demand of guest WLAN - providing fine-grained access control and high security while not putting extra burden on users and administrators. For example, all users (authorized or unauthorized) can join an open guest WLAN. All messages transmitted on an open guest WLAN are in plaintext, and thus users’ private information may be revealed. Meanwhile, it is impossible to realize user access control for an open WLAN. As for password based authentication, though it can prevent unauthorized users joining WLAN, it is hard to revoke visitors’ authorization of WLAN access once they are told the password. This kind of authentication is not secure enough as passwords are guessable. Although the 802.1X authentication server based authentication can provide very fine-grained access control for each user~\cite{ Afia2007Comparative}. However, it introduces a heavy burden on administrators as all users rely on administrators operation on user account. Authenticating by a web portal server equipped by WIFIDog can indeed realize various demand of WLAN access control. However, this kind of authentication cannot provide any extra confidentiality and integrity protection for transmitted data. Data security still relies on the base authentication mechanism, but it is common that this kind of authentication is deployed on an open WLAN which means there is no base authentication mechanism. 


In this paper, we want to combine fine-grained access control, convenience and security for WLAN. Our goal is to achieve differential WLAN access control for visitors and staff - staff can always join WLAN while visitors can only join WLAN during their visit, and dynamic WLAN access authorization for visitors - granting when they come and revoking when they leave. Meanwhile, we do not want to put extra burden on administrators. To achieve this goal, we proposed a location-based evolving passwords scheme for WLAN authentication. WLAN passwords will automatically evolve at regular intervals. Administrators can adjust the update interval whenever needed. If passwords evolve, only authorized users can get new passwords and continue to join WLAN. This means, unauthorized users knowing old passwords will be filtered out and cannot connect to WLAN any longer. To prevent unauthorized users renewing passwords, we introduced physical access control into passwords evolving process. A random number called physical parameter is used to renewing passwords. A specific device generates physical parameters and broadcasts them to a specific location protected by a physical access control system. Thus, users can only obtain physical parameters in constrained locations. In order to obtain physical parameters and renew passwords, authorized users must pass physical access control systems. Visitors who have finished their visit will not be able to get the physical parameter. Thus, they cannot get new passwords and join WLAN, even though they may still receive wireless signal. Most important, the new dynamic password scheme will not put extra burdens on authorized users and administrators. Authorized users can still get passwords from administrators or authorized users and join WLAN as usual. Passwords will evolve automatically without participation of administrators and authorized users. 


We have implemented the proposed scheme. An authorized mobile device can successfully extract physical parameters, calculate new passwords, and authenticate to APs before and after passwords evolving. There is almost no connection delay compared with the static password scheme. What authorized users and administrators need to do is just as usual. 


Contributions:
We proposed a location-based evolving passwords scheme for WLAN authentication, providing fine-grained access control for different users while not putting extra burden on administrators and authorized users. 
%动态权限管理
We combine physical authentication and WLAN authentication: whether users can connect to WLAN depends on whether they can pass physical access control systems. 
%By this way, we achieved dynamic authorization for visitors: %when they come, they need to request for WLAN access %authorization; when they leave, their authorization will be %revoked in time. 
%一般员工WIFI和访客WIFI是分开的,,,,
Regular staff can always pass physical access controls and update passwords. As for visitors, once they ended their visit, they would not pass physical access controls and so they would not get new passwords even though they can still receive wireless signal, so their authorization will be revoked in time
%抵抗穷举攻击和假冒AP攻击
What’s more, we enhanced the security of the password-based WLAN authentication. WLAN passwords are automatically changed from time to time. The password update interval is short enough to avoid brute force attacks and fake AP attacks. Even if an attacker gets a password, he/she will be filtered out when passwords evolve minimizing negative influences of a successful attack. For the fake AP attack, the fake AP can induce the user to access the insecure wireless network by setting the same password. If the password  update regularly,this attack is also invalid.
%大型部署
In our proposed scheme, only one master AP is located in a controlled environment, which produces and distributes physical parameters. Slave aps can be located outside of a physically controlled environment, this mode of operation is suitable for large-scale guest WLANs of large organizations.
