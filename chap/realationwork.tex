\section{Related Works}
Despite a growing number of new authentication mechanisms, passwords remain the dominant method of authentication since they are easy to use, inexpensive to administer, and user-friendly. Authenticated by text-based passwords does not require special devices like fingerprint readers, u keys, and the like. All users need to do is to remember passwords and type them in login box. The authentication process is hardly affected by other factors. For the convenience of memory, users tend to set regular passwords. 


\cite{Nakhila2015Parallel} found there are well marked patterns in passwords after analyzing 32.6 millions of real-life passwords. The password patterns can be categorized into ten. Appending pattern and prefix pattern respectively means digits and punctuation characters are added at the end or the beginning of dictionary words. While inserting pattern mean digits and punctuation characters are inserted into dictionary words. Repeating pattern means choosing certain number combinations, dictionary words, or punctuation characters and repeat them to create a password. Sequencing pattern means passwords are sequences of keyboard layouts, alphabet letters, digits, or their combinations. Replacing pattern means passwords are created by replacing certain letters in dictionary words with a number or a symbol. Capitalizing pattern means passwords are dictionary words some letters of which are exchanged with their uppercase equivalents. Reversing pattern means passwords are dictionary words whose letters are put in a reverse order. Special-format pattern passwords are e-mail address, dates, telephone numbers, and the like~\cite{Chen2015The}. The last pattern - mixed pattern means combining two or more above mentioned patterns. 


\cite{morris1979password} also supports that there are regularities in passwords. It has measured password characteristics on over 6 million passwords, in terms of password length, password composition, and password selection. For password length, it found that the average password length is 9.46 characters, and most passwords have the length between 8 and 10 characters. For password composition, it found that the bulk of passwords consists of alphanumeric characters or only numbers. Furthermore, nearly half of passwords only consists of numbers. Almost all letters in passwords are in lowercase. So it is easy to see that passwords structure still remains simple and vulnerable \cite{Studer2010Mobile}. For password selection, it found that users prefer to use meaningful data in passwords such as Pinyin, English word, year, date, and the like. 


According to the above references, it is evident that though always emphasized, password security still remains a problem. Passwords are simple and easy to guess helping attackers to successfully implement dictionary attacks. 


To improve password security, ~\cite{Yu2017EvoPass} proposed time evolving graphical password scheme. During registration, users need to provide several pictures to the authenticator as their passwords. The authenticator will pick up several similar pictures as decoy pictures. These pictures together with the pictures provided by users will be distorted by a specific distortion algorithm. In authentication, users are required to identify the correct pictures uploaded by users themself. The graphical password can be configured to be strengthened over time. With time going by, all the above-mentioned pictures will be further distorted by applying different distortion parameters. Each time distorted, the graphical password become less recognizable which increases the number of observations required for an attacker to acquire the correct password. However, it has a low impact for users. Users experience the whole and gradual distortion process~\cite{Davis2004On}. With the impression of previous pictures, it is not hard for users to figure out the new distorted password pictures. Therefore, the proposed time evolving mechanism improves the graphical password security without influencing users’ experience. 


Also to improve password security, ~\cite{Ryu2017Location} proposed location related encryption key derivation scheme. All sensitive files are encrypted, but the encryption key is not stored in the local storage. To decrypt encrypted files, users must be at a specific location. A special device is deployed at the specific location and broadcasts random numbers all the time. The random numbers can only be captured in the specific location and is essential for generating the decryption key. That is, users must be at the specific location and capture the random number. Then they can generate the decryption key through several interactions with the special device and decrypt the encrypted files. The location related decryption key derivation mechanism uses the location as the password to get the decryption key and avoids password leakage. Thus, it improves the security of sensitive files. 
